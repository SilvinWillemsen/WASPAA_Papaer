
\section{signal model}
The signal model is derived from the string vibrations, thus we start by modeling the string 
deflection as caused by activation from the guitar pick. 
%
The string has initial displacement $\delta$ at the plucking position $P$, at the
$P$th fraction of its length $(0<P/L<1)$. %At release the vibrating string has length $L$

The well-known equation of motion for an ideal string with length (nut-bridge) $L$ with displacement $y$ and defined over $l \in [0, L]$ (also known as the 1D wave equation) is
%
\begin{equation}
\frac{\partial^2y}{\partial t^2} = c^2\frac{\partial^2y}{\partial l^2},
\end{equation}
%
where $c$ is the transverse wave speed. Assuming that the string is pinned at $l=0$ and $l=L$ (simply supported), its general solution can be written as~\cite{fletcher:physics_of_musical_instruments}:
%
\begin{equation}\label{eq:modalSum}
    y(l,t) = \sum_m C_m\sin(\omega_mt+\phi_m)\sin\kappa_ml,
\end{equation}
%
with frequency $\omega_m = m\omega_0$, wave number $\kappa_m = \omega_m / c$ and amplitude $C_m$ of the $m$th mode with $m = \mathbb{Z}^+$. For a string plucked at the $P$th fraction of its length ($0<P/L<1$), assuming the initial displacement to be triangular as in \cite{fletcher:physics_of_musical_instruments} and initial velocity $\frac{\partial y}{\partial t} = 0, \ \forall l$, $C_m$ can be found using \cite{donkin:acoustics}:
\begin{equation}
    C_m = \frac{2\delta}{m^2\pi^2P(1-P)}\sin(m\pi P), \quad [\cdot].
\end{equation}