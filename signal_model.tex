\input{macros.tex}
\section{signal model}
The signal model is derived from the string vibrations, thus we start by describing the activation from the guitar pick. 
%
The string has an initial deflection $\delta$ at the plucking position, which is at the $P$th fraction of its length $(0<PL<L)$, where $L$ is the length of the vibrating string. At release the displacement $y$ is defined by the 1D wave equation: %
%\begin{equation}
$\frac{\partial^2y}{\partial t^2} = c^2\frac{\partial^2y}{\partial l^2}$,
%\end{equation}
%
where $c$ is the transverse wave speed. Assuming that the string is pinned at $l=0$ and $l=L$, its general solution can be written as~\cite{fletcher:physics_of_musical_instruments}:
%
\begin{equation}\label{eq:modalSum}
    y(l,t) = \sum_m C_m\sin(\omega_mt+\phi_m)\sin\kappa_ml,
\end{equation}
%
where $C_m$ is the amplitude of the $m$th mode, $\omega$ is frequency and $\kappa_m = \omega_m / c$. Assuming the initial displacement to be triangular and no initial velocity $\frac{\partial y}{\partial t} = 0, \, \forall\, l$, the $m$th amplitude $C_m$ is defined by the Fourier sine series~\cite{donkin:acoustics,fletcher:principles_of_vibration_and_sound}:
\begin{equation}
    C_m = \frac{2\delta}{m^2\pi^2P(1-P)}\sin(m\pi P), \quad [\cdot],
\end{equation}
which expresses that the spectral envelope is sinusoidally shaped, dependent on the plucking position $P$ with a scale of $m^{-2}$ for the $m$th mode.
%
\noindent
On the guitar we assume that a pickup is measuring the displacement $y(l,t)$ close to the string at location $l\!\!=\!\!\lambda$. At the discrete time instance $n$ the observed signal $x(n)$ is being sampled, i.e.,
\begin{equation}
     x(n)  \vert_{l=\lambda} \propto y(\lambda, t),
\end{equation}   
where $x(n)$ is the guitar signal recorded with the pickup at $\lambda$. We parametrize $x(n)$ with an inharmonic signal model as proposed in~\cite{hjerrild::icassp19}. 
%
 At time instance $n$, the observed signal vector $\vecx \in \mathbb{C}^N$ is represented as $\vecx = [x(0) \, x(1) \, \cdots \, x(N-1)]^T$, with $T$ denoting the transpose. %Do we need this?
%A complex signal can ease both notation and computational complexity and a real-valued signal is converted to complex by using the Hilbert transform~\cite{LawrenceMarple1999}. 
The $n$th entry of $\vecx$ is modeled as an inharmonic sinusoidal part and a noise part i.e.,  
\begin{equation}\label{eq:sigmod1}
  x(n)\! =  \!\sum\limits_{m=1}^{M}\!\! \alpha_{m} \exp\big({j\psi_m(\omega_0,B) n}\big)+v(n), 
  %x(n)\! = \!s(n)\!+\!v(n)\!= \!\sum\limits_{m=1}^{M} \alpha_{m} e^{j(m\omega_0 \sqrt{1+B m^2}) n}+v(n), 
\end{equation}
where $\omega_0$ is the fundamental frequency, $M$ is the number of partials, $\alpha_{m}$ is the complex amplitude of the $m$th partial, $v(n)$ is noise and the instantaneous frequency  $\psi_m(\omega_0,B)$ is derived in~\cite{fletcher:piano_model} as
\begin{equation}
  \psi_m(\omega_0,B) = m \omega_0 \sqrt{1+B m^2}. 
\end{equation}
For ease of notation, we denote it as $\psi_m$ although it is a function of $\omega_0$ and $B$. The model order $M$ can be estimated~\cite{nielsen2017fast,multipitch}, while for the string model, initialized by the triangular shape in~\eqref{eq:string_initialization} we assume a high $M$ at the onset event. In vector-matrix notation the observed signal is modeled as
\begin{equation}\label{eq:xZa}
  \vecx = \matZ \vecalpha + \vecv,
\end{equation} 
where the complex sinusoidal matrix $\matZ \in \mathbb{C}^{N\times M}$ is given by
\begin{align}
  \matZ =& \left[ \vecz(\psi_1) \: \vecz(\psi_2) \: \cdots \: \vecz(\psi_M)\right], \\
  \vecz(\psi_m) =& \left[ 1 \: e^{j\psi_m} \: e^{j\psi_m2} \: \cdots \: e^{j\psi_m(N-1)} \right]^{T},
\end{align}
where $\vecalpha = [\alpha_1 \: \cdots \: \alpha_M]^T$ is a vector containing complex amplitudes and $\vecv = [v(0) \: v(1) \: v(N-1)]^T$ contains all noise terms. We denote the unknown and deterministic parameters with $\vectheta$, i.e.
\begin{equation}\label{eq:theta_parameters}
  \vectheta = \{\omega_0, B, \vecalpha\}.
\end{equation}
The amplitudes $\vecalpha$ can be estimated with the least squares, while the other parameters $\omega_0$ and $B$ are non-linear. 
The inharmonic pitch and inharmonicity coefficient estimates $\{\widehat\omega_0, \widehat B \} $ are sufficient for classification of string and fret~\cite{barbancho:inharmonicity_tablature,michelson2018_aes} and the estimated amplitude vector $\vecalphahat$ is used for estimation of the plucking position $\widehat{P}$.